%%%%%%%%%%%%%%%%%%%%%%%%%%%%%%%%%%%%%%%%%%%%%%%%%%%%%%%%%%%%%%%%%%%%%%
% LaTeX Template: Curriculum Vitae
%
% Source: http://www.howtotex.com/
% Feel free to distribute this template, but please keep the
% referal to HowToTeX.com.
% Date: July 2011
%Version for spanish users, by dgarhdez
% 
%%%%%%%%%%%%%%%%%%%%%%%%%%%%%%%%%%%%%%%%%%%%%%%%%%%%%%%%%%%%%%%%%%%%%%
% How to use writeLaTeX: 
%
% You edit the source code here on the left, and the preview on the
% right shows you the result within a few seconds.
%
% Bookmark this page and share the URL with your co-authors. They can
% edit at the same time!
%
% You can upload figures, bibliographies, custom classes and
% styles using the files menu.
%
% If you're new to LaTeX, the wikibook is a great place to start:
% http://en.wikibooks.org/wiki/LaTeX
%
%%%%%%%%%%%%%%%%%%%%%%%%%%%%%%%%%%%%%%%%%%%%%%%%%%%%%%%%%%%%%%%%%%%%%%
\documentclass[paper=a4,fontsize=11pt]{scrartcl} % KOMA-article class
							
\usepackage[english]{babel}
\usepackage[utf8x]{inputenc}
\usepackage[protrusion=true,expansion=true]{microtype}
\usepackage{amsmath,amsfonts,amsthm}     % Math packages
\usepackage{graphicx}                    % Enable pdflatex
\usepackage[svgnames]{xcolor}            % Colors by their 'svgnames'
\usepackage{geometry}
	\textheight=650px                    % Saving trees ;-)
\usepackage{url}
\urlstyle{same}
\usepackage{fancyhdr}
\frenchspacing              % Better looking spacings after periods
%\pagestyle{empty}           % No pagenumbers/headers/footers

%%% Custom sectioning (sectsty package)
%%% ------------------------------------------------------------
\usepackage{sectsty}

\sectionfont{%			            % Change font of \section command
	\usefont{OT1}{phv}{b}{n}%		% bch-b-n: CharterBT-Bold font
	\sectionrule{0pt}{0pt}{-3pt}{1pt}}

%%% Macros
%%% ------------------------------------------------------------
\newlength{\spacebox}
\settowidth{\spacebox}{8888888888}			% Box to align text
\newcommand{\sepspace}{\vspace*{1em}}		% Vertical space macro

\newcommand{\MyName}[1]{ % Name
                \Huge \hfill #1
                \par \normalsize \normalfont}
		
\newcommand{\MySlogan}[1]{ % Slogan (optional)
		\large \usefont\hfill \textit{#1}
		\par \normalsize \normalfont}

\newcommand{\NewPart}[1]{\section*{{#1}}}

\newcommand{\PersonalEntry}[2]{
		\noindent\hangindent=2em\hangafter=0 % Indentation
		\parbox{\spacebox}{        % Box to align text
		\textit{#1}}		       % Entry name (birth, address, etc.)
		\hspace{1.5em} #2 \par}    % Entry value

\newcommand{\SkillsEntry}[2]{      % Same as \PersonalEntry
		\noindent\hangindent=2em\hangafter=0 % Indentation
		\parbox{\spacebox}{        % Box to align text
		\textit{#1}}			   % Entry name (birth, address, etc.)
		\hspace{1.5em} #2 \par}    % Entry value	
		
\newcommand{\EducationEntry}[4]{
		\noindent \textbf{#1} \hfill      % Study
		\colorbox{White}{%
			\parbox{5cm}{%
			\hfill\color{Black}#2}} \par  % Duration
		\noindent \textit{#3} \par        % School
		\noindent\hangindent=2em\hangafter=0 \small #4 % Description
		\normalsize \par}

\newcommand{\WorkEntry}[4]{				  % Same as \EducationEntry
		\noindent \textbf{#1} \hfill      % Jobname
		\noindent\colorbox{Black}{\color{White}#2} \par  % Duration
		\noindent \textit{#3} \par              % Company
		\noindent\hangindent=2em\hangafter=0 \small #4 % Description
		\normalsize \par}


\pagestyle{fancy}
\lhead{Curriculum Vitae - aggiornato al 04/05/22}
\lfoot{Autorizzo il trattamento dei dati personali contenuti nel presente curriculum vitae ai sensi del Regolamento (UE) 2016/679.}
\renewcommand{\headrulewidth}{0.4pt}
\renewcommand{\footrulewidth}{0.4pt}
\pagenumbering{gobble}

%%% Begin Document
%%% ------------------------------------------------------------
\begin{document}
% you can upload a photo and include it here...
\includegraphics[width=0.26\textwidth]{Schermata a 2019-01-06 17-49-28.png}

\MyName{Filippo Paganelli}

%%% Personal details
%%% ------------------------------------------------------------
\NewPart{Informazioni Personali}{}
\textbf{Data di nascita:} 03/09/1995 \newline
\textbf{Sesso:} Maschile \newline
\textbf{Nazionalità:} Italiana \newline
\textbf{Indirizzo:} Via Busento, 5/G, Bellaria Igea Marina (RN), 47814 \newline
\textbf{Telefono:} 3461426698 \newline
\textbf{Email:} 
\begin{description}
\item[Personale - ]filippo1paganelli@gmail.com
\item[Università - ]filippo.paganelli3@studio.unibo.it
\item[Lavoro - ]filippo.paganelli@maggioli.it
\end{description} 

%%% Education
%%% ------------------------------------------------------------
\NewPart{Istruzione e Formazione}{}

\EducationEntry{Università di Bologna - Campus di Cesena}{2019 - oggi}{Laurea Magistrale - Ingegneria e Scienze Informatiche - LM32}{}
\EducationEntry{Università di Bologna - Campus di Cesena}{2016 - 2019}{Laurea Triennale - Ingegneria e Scienze Informatiche - L31 \newline Voto: 100/110}{}

\NewPart{Esperienza Professionale}{}
\EducationEntry{DevOps - Ricerca e Sviluppo}{02/2021 - oggi}{Maggioli SpA, Via del Carpino, 8, Santarcangelo di Romagna (RN) \newline Progettazione, sviluppo e manutenzione software/infrastruttura, CI/CD, IaC}
\newline
\EducationEntry{Firmware Developer}{08/2020 - 02/2021}{Mark One s.r.l., Via A. Einstein, 22, Mercato Saraceno (FC) \newline Progettazione, sviluppo e manutenzione firmware per il controllo di stampanti 3D di alta qualità.}
\newpage
\EducationEntry{Tirocinio}{03/2019 - 07/2019}{Università di Bologna - Campus di Cesena, Via dell'Università, Cesena (FC) \newline Analisi del protocollo DTLS e delle comunicazioni real-time peer-to-peer per il ripristino
rapido della comunicazione in caso di cambio IP dei peer.}

%%% Skills
%%% ------------------------------------------------------------
\NewPart{Competenze Personali}{}
\EducationEntry{Lingue Straniere}{}{Inglese - Certificato di Idoneità Linguistica - Università di Bologna - Livello B2 - 30/01/2020}
\newline
\EducationEntry{Software}{}{
\begin{itemize}
    \item \textbf{Sistemi Operativi:} Linux (Avanzato), Microsoft Windows (Intermedio) \item \textbf{Ambienti di Sviluppo Integrato (IDE):} Eclipse (Intermedio), IntelliJ IDEA (Avanzato), PyCharm (Intermedio), Visual Studio Code (Avanzato)
    \item \textbf{Build Automation/Continuous Integration/Continuous Delivery:} Docker (Avanzato), Gradle (Intermedio), Maven (Avanzato), Kubernetes (Avanzato), Helm (Intermedio), OpenShift (Intermedio), GitLabCI (Avanzato), Sbt (Intermedio), Git (Avanzato), make (Intermedio)
    \item \textbf{Cloud Platform:} Azure Microsoft (Avanzato), Amazon Web Services (Base)
    \item \textbf{Infrastructure as Code:} Terraform (Intermedio) 
    \item \textbf{Monitoring Tools:} Prometheus (Intermedio), Grafana (Intermedio), K6 (Base), Azure Monitor/Insight (Intermedio)
    \item \textbf{Linguaggi di Programmazione:} Bash (Avanzato), C/C++ (Avanzato), Java (Avanzato), Python (Intermedio), Javascript (Intermedio), Typescript (Intermedio), Kotlin (Base), Scala (Base) 
    \item \textbf{Framework:} Spring (Intermedio), Kafka (Intermedio), Akka (Base), JADE (Intermedio)
    \item \textbf{Linguaggi di markup/modellazione:} HTML (Intermedio), LaTeX (Avanzato), UML (Avanzato) 
    \item \textbf{Database:} MongoDB (Avanzato), SQL (Avanzato)
\end{itemize}
}
\newline
\EducationEntry{Patente di guida}{}{B, A2}

\NewPart{Certificazioni}{}
\EducationEntry{Introduction to Cybersecurity}{08/2018}{Cisco Networking Academy}
\newline
\EducationEntry{Introduction to Internet of Everything}{08/2018}{Cisco Networking Academy}
\newline
\EducationEntry{BLSD - Basic Life Support and Defibrillation}{10/2015}{Italian Resuscitation Council Foundation (IRC Foundation)}
\newline

\NewPart{Riconoscimenti}{}
\EducationEntry{Borsa di Studio}{A.A. 2018/2019}{Borsa di studio per merito (Università di Bologna)}
\newline
\NewPart{Progetti}{}
\EducationEntry{Approccio Architetturale ad Alto Livello per realizzare DTLS \\ Fast Resumption in WebRTC}{10/2019}{System Integration \\ Studio della libreria WebRTC e delle WebAPIs standard Javascript per l'implementazione di tecniche di ripristino rapido della comunicazione real-time tra peer attraverso il protocollo DTLS (Fast Resumption e Session Resumption).
\begin{description}
\item[Link] \url{https://amslaurea.unibo.it/19075/}
\end{description}}
\newline
\EducationEntry{Forest Fire Detection using Multi-Agent Systems \\
and Wireless Sensor Networks - v1}{03/2020}{Sistemi Distribuiti \\  L'obiettivo di questo progetto è la creazione di un Multi-Agent System che permetta il controllo di una Wireless Sensor Network riducendo il più possibile le dipendenze derivanti da specifici scenari applicativi. Al fine di testare i risultati ottenuti è stato scelto come caso di studio la rilevazione di incendi nelle foreste. L'intero lavoro si concentra sulla progettazione e lo sviluppo del Multi-Agent System tramite l'utilizzo del framework Smart Python Agent Development Environment (SPADE).
\begin{description}
\item[Repo] \url{https://github.com/paganellif/sd-project-paganelli-2021}
\end{description}}
\newline
\EducationEntry{Forest Fire Detection using Multi-Agent Systems \\
and Wireless Sensor Networks - v2}{02/2021}{Smart City e Tecnologie Mobili \\ L'obiettivo di questo progetto è l'estensione del Multi-Agent System per il controllo di una Wireless Sensor Network sviluppato per il corso Sistemi Distribuiti. L'intero lavoro si concentra sulla progettazione e lo sviluppo della parte fisica del sistema, la quale consiste in tre nodi RaspberryPi ed i relativi sensori/attuatori utili alla rilevazione di incendi nelle foreste, ed il deployment in cloud, facendo uso di Amazon Web Services, di tutti i servizi utili, ovvero storage (MongoDB), comunicazione tra agenti (XMPP) e presentazione dati (Web App).
\begin{description}
\item[Repo] \url{https://github.com/paganellif/sctm-project-paganelli-2021}
\end{description}}
\newline
\EducationEntry{Studio Libreria RDFLib e Definizione di una Ontologia RDF/OWL per Wireless Sensor Network e Multi-Agent System}{10/2021}{Web Semantico \\ Il  primo assignment consiste nello studio di una tecnologia riguardante il Web Semantico: la libreria RFDLib. RDFLib è una libreria Python open-source per la manipolazione di RDF, mantenuta su Github e disponibile tramite PyPI. Il secondo e terzo assignment consistono rispettivamente nel modellare la conoscenza di un qualche specifico dominio a scelta (o sottoparte di esso) utilizzando RDF e nell’estensione di quest'ultimo utilizzando OWL. Il dominio scelto è quello  rappresentante i progetti già realizzati per i corsi \textit{Sistemi  Distribuiti} e \textit{Smart City e Tecnologie Mobile}: Forest Fire Detection using Wireless Sensor Network and Multi-Agent System.
\begin{description}
\item[Repo:] \url{https://github.com/paganellif/ws-project-paganelli-2021}
\end{description}}
\newline
\EducationEntry{FRP-Scala: Studio e sperimentazione del paradigma Functional Reactive Programming in Scala}{01/2022}{Paradigmi di Programmazione e Sviluppo \\ Lo scopo del progetto individuale consiste nello studio e nella sperimentazione del paradigma Functional Reactive Programming (FRP) in Scala. Il primo capitolo tratta lo studio delle principali caratteristiche del paradigma FRP, derivanti dall'unione dei due paradigmi funzionale (FP) e reattivo (RP), e della architettura di una applicazione FRP. Successivamente vengono analizzate e confrontate alcune librerie open-source per il paradigma FRP in Scala, scelte fra quelle disponibili al momento dello svolgimento del progetto, sviluppando alcuni semplici esempi di test per sperimentare pro e contro delle librerie. Nell'ultimo capitolo si descrive come è stata progettata e sviluppata una semplice rivisitazione del videogioco Snake utilizzando il paradigma FRP, il linguaggio Scala e una fra le librerie individuate nel capitolo precedente.
\begin{description}
\item[Repo:] \url{https://github.com/paganellif/pps-project-paganelli-2122}
\end{description}}

\end{document}
